\chapter{Introduction}
\section{Objectifs}

Dans ce rapport, nous étudions une approche automatique et robuste pour la détection, segmententation et classification des objets urbains sur des nuages de points 3D via un scanner laser ainsi qu'une imagerie digitale par morphologie mathématique et apprentissage supervisé. 

Le traitement est effectué via des images d'élévation, appelées \enquote{modèles numériques de terrain} et permet une projection du résultat final sous forme de nuage de points 3D. 

\section{Organisation}

Nous suivrons la structure de l'article, en reprenant dans un premier temps, l'état de l'existant, puis l'étude des méthodes permettant le traitement d'images 3D, l'étude comparative par rapport à l'existant, et enfin, une conclusion.